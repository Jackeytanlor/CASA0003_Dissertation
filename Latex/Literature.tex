\section{Literature review} 


\subsection{Sustainable Development}
In 2015, the United Nations Development Program (UNDP) formed 17 global goals known as “Sustainable Development Goals” with 169 targets and 232 indicators for the protection of the planet for current and future generations \parencite{pedersen_zari_ecosystem_2012}. According to SDGs, Sustainable Development Goal 15 (SDG15) would be aimed at protecting, restoring and promoting sustainable use of terrestrial ecosystems \parencite{un_transforming_2015}. Therefore, when it comes to the urban fringe area development, it should be basically focused to cope up with social, economic, and environmental benefits in the present scenario. In order to achieve SDG15, it should be focused on three perspectives equally \parencite{howlader_exploring_2020}. Therefore, how to balance the social, economic, and environmental benefits in urban development has become one of the most complicated problems facing governments worldwide.\\

\subsection{Interventions to urban fringe area}
Since there has been a fear that farmland would be swallowed up by urban sprawl with the growth of megacities, it is generally considered that it is necessary to think the need for possible physical intervention in the urban fringe area \parencite{gallent_representing_2007}\parencite{gallent_ruralurban_2006}. It is considered that the intervention method should be built on a wider urban agenda concerned with growth and the management of growth. By considering this, what form should interventions take to manage urban fringe areas has been generally discussed in different countries and megacities. \\

Green Belt policy was used as a universal solution to urban growth by thinking of urban fringe area \parencite{gant_land-use_2011}. In London, intervention in urban fringe area would be fulfilling the function of fire break to protect the environment. The Green Belt has been a prime part of the land-use planning system for planners and could distinguish the area of urban and rural land use. Apart from the function of butter zone in ecological space, the Green Belt also provide opportunities for outdoor recreation near urban areas to citizens, retains and enhances attractive landscapes, and most importantly, secures nature conservation interest as well land in agricultural, and forestry uses \parencite{ferguson_informal_1979}. \\

The Green Belt policy successfully made urban containment in spatial distribution. However, \textcite{gant_land-use_2011} think that the intervention in the past has ignored the possibility that urban fringe area might have a varied character worthy of close attention. Although urban planners realized that they had made a mistake of seeing fringe area as buffers and tried to relax Green Belt retractions, the urban fringe area was still disconnected to nearby urban and rural areas. Therefore, the Green Belt policy should recognize the strategic needs of public service development and the importance of rural-urban fringes to view the urban fringe area in a wider subregional and regional context \parencite{gallent_representing_2007}.\\

Furthermore, in the process of exploring sustainable development in urban fringe area, the Netherlands was treated as the success of current open space preservation policies. Green Heart planning, establishing a green spaces network within the city to slow down urban expansion in the form of green spaces combined with surrounding buffer zones, was developed in the Hague Region, one of the most urbanized areas in the Netherland according to national spatial plan\parencite{koomen_impact_2013}. Apart from adopting a buffer zone policy, the government also assign the green infrastructure from urban fringe area to the status of a municipality or assign land ownership and stewardship to a community land trust \parencite{aalbers_analysis_2009}. A unified system of management at a regional level provides greater clarity in the allocation of resources. Moreover, by developing glass and grass production and recreation area, it could successfully make market chains and urban-rural relationships compared with developing housing site in other cities. Therefore, urban fringe area in Netherland would be in a place where recreational facilities and natural areas were being developed by controlling their dynamic balance \parencite{koomen_impact_2013}.\\

It has been considered that urban fringe areas are a distinct entities because of their special characteristics and productive construction in each region \parencite{gallent_ruralurban_2006}. Many research indicated that sustainable intervention of urban fringe area in developing countries would be different. Researchers also attempted to introduce the SDGs15 in the fringe areas so that it could draw sustainability in metropolitan cities. However, it is hard to achieve the SDGs15 goal of restoring degraded forests and substantially increasing afforestation of ecological space in Indian scenario of urban fringe area in fast-growing megacities\parencite{howlader_exploring_2020}. With the rapid growth of population and environmental pressures in urban area, urban fringe area development is mainly focused on decreasing and decentralizing the pressure of the central area. Treated as a core sub-centre in the future, Indian megacities are other priorities on hand compared with sustainable development. Therefore, they are more likely to form the Development Authorities (DA) to adopt and implement integrated policies and plans toward inclusion.\\

There have been a handful of attempts to address urban sprawl in China. Megacities in China sprawled most from 2000 to 2010, and the rate of urban sprawl has decreased since 2010 \parencite{liu_urban_2018}. The Land Administrative Law and the Regulations on the Protection of Basic Farmland are promulgated to implement open space preservation. Specifically, urban sprawl would show an obvious difference in megacities depending on region, population, and administrative hierarchy \parencite{li_urban_2019}. \\

Different cities would consider the difference and formulate effective regulatory policies in the urban fringe area. In the context of rapid and large-scale city construction, scholars in China suggest local governments should enhance their control and propose local planning to serve the needs of growth \parencite{tian_measuring_2017}. For example, according to the planning outline of ecological civilization construction in Guangzhou(2016-2020), important green corridors and nodes companies with two ‘green forest rings’ with a total area of 86 km2 are proposed by Guangzhou environmental protection bureau to mitigate environmental pollution from urban sprawl(Guangzhou environmental protection bureau, 2016) \parencite{yu_spatial_2007}. Besides, according to Shanghai City Master Plan (1999–2020), Shanghai from a development strategy, “One City, Nine Towns”, to alleviate the city from the significant pressure of urbanization \parencite{tian_measuring_2017}. With a projected population of 800,000 to one million in each new town, Shanghai also transferred local land revenues and land use from the municipal government to district governments to ensure the efficiency of environmental reservation \parencite{wang_dynamics_2020}. In conclusion, Although the intervention strategies have achieved some success in China, urban development strategies in fringe areas are still imperfect due to their late start and the dominant power for urbanization. Therefore, more research should be applied to quantify the impact of planning on urban fringe area and sustainable development planning could be explored in the future based on the result.\\



\subsection{Research technique}
\subsubsection{Identification of urban fringe area}
Traditionally, it would be common to identify urban fringe area by using statistical analysis methods \parencite{beibei_review_2012}. By considering single or multiple indicators including density, population, economic level and land use, research might also use Multicriteria Analysis (MCA) to combine each indicator and finish the identification process \parencite{yang_spatial_2017}. Due to the limitations such as uncontinuous statistical data and the difference of statistical standards, it would be difficult to obtain efficient and accurate results.\\

To minimize the uncertainty, a combination of multi-date Landsat Thematic Mapper (TM) with different years was used to detect urban detailed urban land use \parencite{turker_land_2005}. However, accurate and timely urban data would be difficult to obtain since the long-time processing and interpretation of certain remote sensing images \parencite{li_urban_2019}. Nighttime light data (NL) could be another selection from much research. For example, \textcite{feng_using_2020} has used DMSP/OLS nighttime light data to identify fringe area. By detecting weak near-infrared radiation, It could be a good way to search urban spatial patterns, and human activities and also recognize the ecological environment and other fields \parencite{bennett_advances_2017}. However, only using nighttime data might accurately estimate spatial patterns in advanced countries but perform less accurate in developing economies \parencite{zhang_can_2013}. Considering this, spatial cluster analysis could be an improved method for the identification. Comminating the K–means algorithm and nighttime light data, it would be more likely to find more details related to urban fringes when compared with the only indicator of nighttime data identification. The method was also commonly used by scholars because of its convenience and versatility.\\


\subsubsection{Indicators of socio-economic development}
When working with complex socio-economic development system in spatial pattern, researchers are faced with the challenge of translating a development scenario in the different area into quantitative evaluation \parencite{zhou_nighttime_2015}. Descriptive statistics are still one of the most important references for the study of urban area indicators, for example, gross domestic product, electric power consumption, and population density will serve as a landmark indicator to measuring the balance between socio-economic development and urbanization from the perspective of government \parencite{zhao_tweets_2018}. However, these data cannot evaluate the socio-economic development at a fine spatial scale, since the socio-economic data would be more likely to be collected and released as an administrative unit, such as provincial, city, and county levels \parencite{peng_integrating_2020}. Therefore, there would be no way to research socio-economic changes in a smaller scale(such as urban fringe area).  \\

One of the widely discussed aspects of urban development could be related to the physical structure or urban form of the cities. The term urban form and structure covers geometric shape, land use and infrastructure. Besides, implications for indicators fragmentation could also be considered \parencite{dominik_wiedenhofer_energy_2013}. Since land use types in the suburban area have been replaced gradually by more efficient land use types in construction land, the area proportion of construction land is widely used to represent land development intensity of the area. For example, Guangzhou Spatial Planning(2020-2035) has clearly stated that the area of land development with construction land as the standard should be controlled to decrease the rapid expansion of the urban area. It means that the area of construction land could be a reference for measuring the urbanization process and socio-economic status of the city. \\

On the other hand, for a given amount of construction land, land use patterns are more likely to be compact in the urban area, while in the rural area construction land could be more scattered \parencite{zhang_can_2013}. Therefore, with the intensified phenomenon of urban sprawl from megacities, The level of accessibility and commercial development between urban areas also decreases as the distance from the city center becomes greater, especially scattered construction land. What is obvious is that studying the area of construction land simply could not Accurately quantify the land development intensity of the region. Therefore, compact development significantly contributes to the idea of ecological and sustainable high-intensity development, it has been popularized as a quantitive measurement \parencite{rahman_gis-based_2022}.\\

Apart from measuring urbanization process by using urban form indicators including area of construction land and compactness, socio-economic indicators should be used to evaluate the economy of whole construction land. Nighttime light data is still a widely applied tool in quantitatively evaluating socioeconomic systems over large areas because of its efficiency and economy \parencite{zhao_tweets_2018}. Many researchers investigate also that there has been a correlation between nighttime data and GDP, population density and electrical power consumption \parencite{zhou_nighttime_2015}. It is observed that VIIRS data from the nighttime dataset could perform better than DMSP-OLS data in estimating socioeconomic parameters on multiple scales \parencite{zhou_nighttime_2015}.\\


\subsubsection{Indicators of environmental outcome}
One way of exploring the environmental outcome of urban sustainability is to use 'ecosystem services' to measure pollution and other environmental problems. For example, carbon fixation, biodiversity and habit quality can provide a view of ecosystems from the perspective of ecosystem support, regulation and provisioning \parencite{phillips_evaluating_2008}. \\

Within the last few decades, The functionality of remote sensing technology in mapping the environmental monitoring of ecosystems has increased considerably.It is commonly considered that monitoring long-term ecosystem changes in trend components by using remote sensing technology could be significant for a better understanding of change in ecosystem's trajectories in ecological space \parencite{zewdie_monitoring_2017}. Therefore, when studying environmental outcome, using parameters from multiple perspectives of the ecosystem and long-term data can be concerned with more dynamics and details of ecosystem changes.\\

Net primary productivity (NPP) is a main aspect of the global carbon cycle, which represents the rate of fixing CO2 from the atmosphere vegetation in an ecosystem \parencite{jiang_modelling_1999}. Consequently, it could be commonly used as indicators to characterize vegetation vigour \parencite{hazarika_estimation_2005}. In particular, spatial patterns of terrestrial NPP could be expected to change when it comes to the cause of human-induced alterations in climatic conditions or other factors. In the context of climate change, more carbon should be sequestered in ecosystem. Thus, the selection of carbon fixation could be of increasing interest worldwide, which would be used as the evaluation of ecosystem.\\

Due to changes in vegetation density, the quantity of light absorbed by the plant canopy can be affected by the changes in vegetation. Therefore, as one of the major indicators, the normalized difference vegetation index (NDVI) historically could be used as a surrogate for ecosystem productivity or general ecosystem energy with a strong correlation between photosynthesis and biomass \parencite{phillips_evaluating_2008}.\\

Specifically, NDVI in time series could also evaluate ecosystem dynamics when It comes to the cause of vegetation transitions and climate change \parencite{zewdie_monitoring_2017}. In addition, because of the strong correlation between vegetation cover gradients and species richness, NDVI can be further used as a quantitative indicator for measuring biodiversity in ecosystem service.\\

On the other hand, studies have analysed the impact of green infrastructure on urban ambient air pollution and thus the ecological environment in terms of direct urban sprawl \parencite{bonilla-duarte_contribution_2021}. Airborne particulate matter (PM2.5) is one of the most important environmental problems, which has significant negative effects on human body \parencite{wu_effects_2018}. In contrast to other ecosystems, PM2.5 is a direct quantitative indicator of air pollution. Therefore, PM2.5 is a good way to evaluate the environment from the point of view of human health.\\
