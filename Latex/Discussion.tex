\section{Discussion}

\subsection{Research Implications}
This study provides a method for quantitative evaluation of fringe areas. This k-means clustering and edge detection method could allow the identification of cities area and bulk analysis of each urban fringe area.\\

The study can also discover urban development trends by using quantitative analysis of urban expansion mode in the past 9 years. In the process of studying data trends, the research would review relevant policies, urban development, and green space spatial planning policies over the years, which can search for effective evaluation of policy strengths and weaknesses. It can also Identify relevant problems with specific indicators and provide recommendations for future development policy.\\

Besides, through two case study cities (Guangzhou and Shanghai), the study could explore the development trend of a megacity in developing countries, especially in China. Specifically, there are advantages of economic urban development in coastal cities. Its ecological environment will also have considerable ecological advantages due to its water yield and other ecosystem services. At the same time, the coastal factor also has the necessary conditions to promote economic development by trade ports. \\

It is during this time that Guangzhou and Shanghai have been developed effectively. According to Guangzhou spatial planning (2020-2035) and Shanghai master planning (2020-2035), the development of the industrial development belts along the river has been consciously promoted. Guangzhou is planned to set a sub-center in the southern part of the city near the South China Sea. It has also consciously set the goal of building an international shipping hub to improve the function of international commerce. Shanghai, on the other hand, is planning to develop a new port area in the southeastern part of the city near the West China Sea. These planned areas are both located within the urban fringe. \\

Taking Guangzhou as an example, the strategy for the protection and management of the green space system in the spatial plan has been proposed to improve the environmental level of the northern area, optimize the quality of the central green space and increase the ecological space in the south area. Many scholars have to consider how much ecological green space should be added to the southern part of the city while considering the current situation of developing port trade.\\

By observing the fluctuation of ecological space and urban development space in urban fringe area, the study can effectively discover whether the environmental outcome of urban fringe area has increased or decreased within the suitable interval in the past years. Thus, it can effectively support the city planner in planning the ecological space and provide a reference for how much ecological space needs to be increased in the southern part of Guangzhou, for example. It also helps the megacities policymakers to quantify whether to protect all aspects of green space in the region when formulating development policies. This also helps them to coordinate ecological resources such as rivers, woodlands, and farmlands.\\



\subsection{Exploration of area changes}
\subsubsection{Socio-economic development: from expansion to contraction}
Through the result of 3 urban development indicators from urban fringe area in Guangzhou and Shanghai, the study found that the general growth rate of CL and AL have been decreasing, while the growth rate of NT has increased sharply in the last three years. It could be seen in general that the level of urban development has raised but the rate of expansion of urban area has decreased. It could be due to the fact that the urban area has gradually changed to be concentrated. Research shows that the aggregation of existing constructed land resources for urban development, which could successfully avoid the inefficient use of land \parencite{wang_dynamics_2020}. Overall, this development trend is positive. The change in CL and AL growth rates indicated that urban sprawl was gradually being halted. What’s more, the development pattern of the city in the last three years has also started to develop into an increase in commercial vitality and facilities within the built-up land. This could be also reflected in the significant increase in NT values over a nine-year period.\\

The development strategy(Shanghai City Master Plan (2020–2035)) has slowly changed from an urban sprawl-based development to an optimized land and space mode. For example, Guangzhou has removed the growth target of construction land from the plan and proposed a 15-minute walking distance concept. It also improved the public service facilities and built community centers to complete the spatial gathering in commercial facilities and population.\\

Nevertheless, the rate of urban constructed land conversion in the urban fringe remains at a high level and the concentration of urban area remains low. The data in the urban fringe area in the result shows that the development of urban fringe area was still driven by the expansion of building land. Despite reasonable planning under existing policies, sustainable development progress remains slow. Therefore, it is important to continue to optimize the direction of urban development by using additional strategies.\\


\subsubsection{Environmental outcome: decreasing ecosystem and optimizing air quality}
Surprisingly, it could be found through result that the environmental index has been increasing in the urban fringe area from 2013 to 2019. When the study is refined to the changes of 3 indicators in environmental system, we could clearly see the decrease in the overall rate of ecological degradation(NPP and NDVI). It could also prove that ecosystem service remained on an overall downward trend over the nine years. \\

With the guidance of the green space policy, megacities are doing their best to protect the ecological environment while developing the socio-economic status. Following the planning outline of ecological civilization construction in Guangzhou(2016-2020), Guangzhou could well regulate the air quality and ecological environment supported by the surrounding ecological resources to improve the current situation.\\

The strategy of greening the environment was effective, but it was still difficult to contain the ecological changes in the fringe area. The overall quality of air quality has increased. It was the improvement in air quality that effectively saw the overall environmental index rise over a 9-year period. It showed that cities were making progress on carbon emission policies and were able to optimize air quality. According to Shanghai City Master Plan (2020–2035), the planning strategies were going to create a high-level low carbon island concept in the northern islands of the region. However, the local low-carbon policy attempts were still not effective in improving the urban fringe area, and it may be necessary to implement low-carbon policies on a smaller scale in more areas.\\

However, when it comes to the difference between fringe area and urban area, the rate of air quality optimization was still lower than in the urban area, especially in Guangzhou. In the urban fringe area in the east and west of Guangzhou area, the PM2.5 concentration was still the highest. And the degree of change in this region has remained in the state of slow change in 9 years.\\

It could probably be due to the construction of industrial areas in the urban fringe on the one hand, and the shift of industries in the urban area on the other hand, which can lead to an industrial structural disadvantage in terms of air quality in the urban fringe area.\\



\subsection{Future optimization and restoration from the research: urban land use consolidation and ecological protection}
By comparing urban development system and environmental system in two cities, the study shows that urban development patterns still exist in the area with the cost of the ecological environment, especially in ecosystem services including carbon and water resource. With their relatively limited ecological resources, Guangzhou and Shanghai still need intensive ecological conservation in the direction of conservation. Therefore, optimization and restoration strategies in the urban fringe area should be purposed when facing this problem. \\

Two case study cities have carried out a series of projects and plans such as ecological protection redline, however, these “bottom-line” policies mainly concerned ecological space while neglecting the human-land contradiction in urban fringe area \parencite{xu_developing_2021}. An inefficient construction land reduction plan and land levelling plan were also purposed by city planners in the Master plan from Shanghai. However, the plan still has no specific unified implementation strategy \parencite{sun_road_2021}.\\

The study would propose some methods for land use consolidation and ecological protection according to the result from two case study cities. When it comes to urban development system in urban fringe area, the land use consolidation method should be implemented. For example, fragments patches should be Integrated through land consolidation, which can integrate different spaces with various kinds of land-use area in a unified way \parencite{bush_building_2019}. It can be carried out spatial reconfiguration in the planning process in the district area. Since the planning of land functions would take the district as the planning unit, the district planning should start from a centralized planning approach, which means that the construction land should be unified into one community in planning.\\

Ecological restoration in urban fringe area should be a conservation and restoration priority. For example, constructed land in the urban fringe area could be improved by combining green and blue spaces to form an ecological network and green infrastructure \parencite{yu_critical_2020}. Small green spaces could be constructed and distributed throughout constructed land to maintain the steady of ecosystem. Among them, in Shanghai short-term territorial spatial planning (2020-2025), urban planners have made a preliminary plan for the short-term ecological pattern of 'one green ring, 3 ecological corridors and 4 ecological park areas'. Where the green ring was to be built around the urban construction area in the central part of the city. And the plan has established two ecological corridors in the area of the urban fringe in the south of the city. This ecological corridor was connected with green areas along the road. The ecological corridors could be restored and constructed by the environmental outcome of this study. The specific way of restoration could be in the form of small green space. On the other hand, ecological restoration in urban fringe area could also improve by enriching the urban fringe corridors with various types, vegetation species and vertical structures\parencite{yang_construction_2018}.\\


\subsection{Limitation and challenge}
Due to the limited time of the dataset, there is no way to analyze the study in more depth from a 10 year to 20 year perspective. According to some research, factors such as road traffic could be considered in the identification of urban fringe areas. Therefore, more influential factors need to be taken into account in future studies.\\

On the other hand, due to the small variation in some of the indicators, there could be no way to effectively quantify the valid results of the indicator. In the future, a selection of case cities in different policy contexts (protection of development priority) could be considered for more in-depth analysis.\\
