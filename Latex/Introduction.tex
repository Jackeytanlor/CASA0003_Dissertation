\section{Introduction} 


\subsection{The vulnerability of ecosystems and climate change}
Climate and environmental change are one of the hottest topics of the 21st century. Many scholars believe that the stability of the ecosystem would be gradually decreasing under the disturbance of human activities, which might eventually have an impact on the ecological environment and the stable development of the city. Therefore, the study of the vulnerability of ecosystems and climate change has been a significant research direction in most ecological and environmental studies during the past several decades. The changes in ecological environment could threaten to shift vegetation, disrupt ecosystems, reduce biodiversity and even damage human well-being \parencite{gonzalez_global_2010}.\\

The panel on Climate Change (IPCC) pointed out that global greenhouse gas emissions rose by 70 percent due to human activities in the context of climate change topic from 1970 to 2004 \parencite{programme_buildings_2009}. Many experts considered that with the rapid development of urbanization, the world may experience potentially dangerous in climate and environmental change. It could have a significant impact on our environment, economies, and societies \parencite{graham_building_2009}.\\

\subsection{Urban sprawl}
At the same time, the phenomenon named “urban sprawl” has also emerged in many countries, which has become a major concern because of its detrimental effects on a series of ecological, economic and social issues \parencite{brueckner_urban_2001} \parencite{jaeger_suitability_2010}. Urban sprawl could be specified as the spilling over of urban-type buildings and construction land into the suburban and farmland areas and the disorganized growth of settlements in farmland areas \parencite{wackermann_handworterbuch_1968}. It would result in encroaching excessively on farmland, leading to a loss of amenity benefits from open space as well as the depletion of farmland resources \parencite{brueckner_urban_2001}, which could also lead to air pollution due to the long commutes generated by urban expansion \parencite{fenger_urban_1999}. Therefore, the social-ecological problem caused by urban sprawl should not be ignored by governments. With ecological space becoming scarcer at an alarming rate, much higher efforts are necessary to conserve and properly use land and soils resource \parencite{haber_energy_2007}.\\

According to Demographia World Urban Areas, 15 of the world's 20 largest cities are in developing countries \parencite{demographia_world_demographia_2021}. It means that developing countries have become the main driving force of global urbanization. In order to meet the needs of economic development, cities in developing countries often require large-scale land development. It might raise the issues of human-land conflict and urban-environmental conflict when it comes to land use transformation. And thus these cities would be the areas with the most intense urban sprawl conflicts \parencite{yue_measuring_2013}.\\

At the same time, Urban sprawl in China has also attracted general concern from scholars. There has been rapid and unprecedented urbanization in China. For example, 51.27\% of China’s entire population now live in cities while 75\% of Chinese would estimate to be in cities within 20 years. It may result in accelerating drastic urban sprawl over almost all of the last decade \parencite{li_urban_2019}. However, due to the difference in urban form between China and other countries, there is relatively little research about the impact and future trends of urban sprawl in China compared with the rich literature on urban sprawl in developed countries \parencite{wang_dynamics_2020}. In addition, with policies supporting industrial development, the industrial-oriented growth model of China's cities in the past has tended to result in a lack of necessary land control. Consequently, excessive urbanization has put enormous pressure on the ecosystem. At the same time, without relevant research in China, there is no way to have a scientific and systematic guideline to deal with the urban sprawl problem in a structured way. Therefore, it is necessary to research urban problems following a dynamic analysis of socio-economic, and ecological aspects of sprawl.\\


\subsection{Urban fringe area}
The urban fringe area could refer to the outer zone of the urban built-up area, which has unique cross-over characteristics \parencite{cui_construction_2020}. Generally, urban fringe area is facing urbanization-related social-ecological problems, especially in megacities in the background of urban sprawl \parencite{peng_integrating_2020}. There would be a discussion about pro-growth or anti-growth interests in many urban fringe areas \parencite{pacione_development_1991}. However, it is popular to decentralize the population by minimizing the growing development pressure of the metropolis' emergence into urban fringe area \parencite{howlader_exploring_2020}. With the growth of urban population and the expansion of industries, there is no doubt that the expansion of construction land can maintain a continuous rise in the socio-economic status of the city. However, the development of construction land represents a reduction in ecological space and farmland in the urban fringe area. At the same time, all energy and material resources are used to build and operate buildings would also cause the growth of greenhouse gas, impacting the supply of ecosystem services \parencite{pedersen_zari_ecosystem_2012}. Therefore, there would be a trade-off relationship between urban growth and environmental change, and between urban growth and socio-economic growth. \\

Although the urban fringe area is challenged with social-ecological problems with the urbanization process, there is no standard principle for researching urban fringe area due to its complexity, dynamicity and fuzzification \parencite{dong_method_2022}. Accurately identifying the urban fringe can significantly help to integrate urban-rural development planning in megacities \parencite{peng_integrating_2020}.\\


\subsection{Research objective and research questions}
How can megacity balance the socio-economic development and environmental outcomes in urban fringe area?\\

The study aims to review existing spatial identification research and select widely applicable identification methodology in megacities to identify urban fringe area. The study also explores the data available and uses the dataset to aggregate socio-economic development and environmental outcome to explore the dynamic changes between urbanization and ecological space in urban fringe area.\\

Apart from the identification and assessment, the study would also attempt to discuss social-ecological problems in depth from a policy perspective by combining ecological restoration and regional management approaches, which can also explore the special characteristic of proper intervention policies in developing countries, especially in China.\\

(1) The study may compare results and existing urban fringe area management policies of different cities.\\ 
(2) The study also discover what kinds of interventions (development priority or protection priority) should be taken in urban fringe area in different case cities.\\

Through objective about, the research would provide a scientific reference to urban growth management in specific megacities. 

