\section{Conclusion}
Urban construction and socio-economic development remain the dominant theme of development in developing countries as well as all over the world. However, when it comes to the context of climate change and urban sprawl, the question of how to develop the city sustainably in the face of unrestricted urban growth and population expansion is something that needs to be deeply discussed by a wide range of scholars. The urban fringe area is the most prominent area for urbanization-related social-ecological problems. The urban fringe area is the most prominent area for urbanization-related social-ecological problems. As the most mature area of urbanization, megacities should develop this area through a well-coordinated and balanced approach.\\

The study explored the trends of urban development system and environmental system in urban fringe area through two case cities in China, Guangzhou and Shanghai, and tries to find the balance point between the two systems.\\

Specifically, the study identified the urban fringe area in two cities by K-means clustering analysis. The study also investigated the relationship between urban fringe area and total area and compared the urban fringe area with the trend of urban development and ecological changes in the past 9 years to explore the effectiveness of the fringe area under the current policies.\\

Through K-means analysis, the study effectively identified urban area, urban fringe area and suburban area. Compared with other areas, the urban fringe area has changed more significantly in urban development system and environmental system in 9 years. Besides, the land use type of this area had become more fragmented.\\

The data analysis of urban fringe area in the two cities shows that urban development index and environmental index of Guangzhou and Shanghai have been on an increasing trend from 2013 to 2019. The urban sprawl has gradually slowed down, and the urban strategy has gradually changed from external expansion to internal spatial optimization. In terms of environment, the air quality of Urban fringe area has improved significantly, but ecosystem services such as carbon fixation still show a decline in general.\\

In response to the result, the study tried to propose a development strategy for urban fringe area with the integration of land-intensive development and ecological restoration. The study also explored the method to integrate the scattered construction land into a uniform large-scale patch in the actual planning and construction process and convert the useless and low-value construction land into ecological space. By doing this, the level of urban development could be improved. In addition, ecological restoration strategies such as increasing scattered green space and optimizing ecological corridors could be used to protect and utilize ecology. \\

This study aims to help megacities to evaluate urban fringe area policies by proposing a unified and convenient method for identifying urban fringe area. Besides, it can also quantify the socio-economic and environmental evaluation of urban fringe area. The project helps policymakers to evaluate the policies of urban fringe area. It will provide a reference for megacities' development model in sustainable policies.\\
